\documentclass[a4paper,10pt]{report}
\usepackage[utf8]{inputenc}

% Title Page
\title{PWM Hub: User Guide}
\author{Federico Balaguer}


\begin{document}
\maketitle

\section{Introduction}
PWM Hub is a source-base library that makes it possible to produce several PWM signals. 

Pulse-Width Modulation is a technique used to encode a message into a pulsing signal. It is a technique commonly used on microcontroller programming. A PWM solution usually is evaluated given two aspects: number of pulse outputs and precision of the duty-cycle. 

Microcontrollers usually include a module that implements PWM. A PWM module sometimes comes short as a solution because of two reasons. First, it implements only a pair of pulse outputs. Second, the precision of the duty-cycle are bound to the internal oscillator settings.

PWM Hub is a library that implements PWM module by software that makes it possible to have a number of pulse outputs and it decouples the precision of the duty-cycle from the frequency of the internal oscillator.

The PWM signal has the following parameters:
\begin{description}
  \item[Period] It acts as a reference to the output signal, the output signal will be on high state at most the period time.     
  \item[Output Signal] It has two possible states: High and Low. It represents a given datum as a high state time. 
  \item[Duty-cycle] It is the relationship betwen Output Signal high state time and the Period.
  \item[Dead bandwidth] It is the minimum time that represents different data.
\end{description}

A problem araises when the system uses high frequency oscillator fuse and the PWM spec has a small Period.   


\section{Instalation}

\section{Configuration and Use}

\end{document}          
